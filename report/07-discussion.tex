\chapter{Discussion\label{chap:discussion}}
This project successfully demonstrates the feasibility and effectiveness of integrating differential privacy techniques into a high-performance data processing pipeline using Fluent Bit and WebAssembly (WASM). The development of the Rust-based filter plugin showcases the potential for leveraging modern programming languages and technologies to create portable, secure, and efficient solutions for privacy-preserving data analysis.

The choice of Rust as the implementation language proves advantageous due to its strong type system, memory safety guarantees, and extensive ecosystem of libraries and tools. The use of the RV crate for generating noise from probability distributions and the serde crate for serialization and deserialization simplifies the code and ensures well-tested and optimized functionality for key operations.

Compiling the plugin to a WASM target enables seamless integration with Fluent Bit's plugin system, allowing for the execution of the differential privacy mechanisms in a sandboxed environment. This approach ensures portability across different platforms and enhances security by isolating the plugin's execution from the host system.

The flexible configuration system, based on TOML files, allows users to easily specify and adjust privacy settings for each data stream. This adaptability is crucial for accommodating diverse data sources and privacy requirements in real-world scenarios. The plugin's modular design and error handling mechanisms contribute to its robustness and maintainability.

The evaluation of the plugin using sample datasets highlights the trade-offs between privacy and utility inherent in differential privacy. The results demonstrate that the Laplacian and Gaussian mechanisms, with carefully selected privacy parameters, can effectively protect individual privacy while preserving the overall statistical properties of the data. The error metrics and data visualizations provide insights into the impact of noise addition on data accuracy and distribution.
However, it is important to acknowledge the limitations and challenges associated with differential privacy. The choice of privacy parameters, such as $\epsilon$ and $\delta$, requires careful consideration and may depend on the specific data and analysis requirements. Balancing privacy and utility remains an ongoing research problem, and the optimal settings may vary across different domains and applications.

Moreover, the composition of differential privacy guarantees across multiple queries or analyses poses challenges in terms of privacy budget management and the accumulation of privacy loss. While this project tracks the cumulative privacy loss by summing the $\epsilon$ and $\delta$ values for each release, more advanced composition theorems and granular per-user budgets could be explored to provide tighter bounds on the overall privacy guarantees.

The integration of the filter plugin with Fluent Bit demonstrates the potential for incorporating privacy-preserving techniques into existing data processing frameworks. However, further work is needed to assess the scalability and performance of the solution in large-scale, real-world deployments. Additional optimizations, such as parallel processing and efficient memory management, could be investigated to enhance the plugin's efficiency and throughput.

Future research directions include expanding the range of supported probability distributions and differential privacy algorithms to cater to a wider variety of data types and analysis tasks. Automated sensitivity analysis could simplify the configuration process and reduce the risk of misconfiguration. Integration with other log processing systems and data analytics platforms would increase the plugin's versatility and adoption.

Furthermore, enhancing the evaluation framework with more comprehensive and automated tests, as well as support for user-provided datasets, would facilitate the assessment of the plugin's performance and correctness. Incorporating advanced data visualization and statistical analysis tools into the evaluation pipeline could provide deeper insights into the impact of differential privacy on data utility and inform the selection of appropriate privacy parameters.

In conclusion, this project contributes to the field of privacy-preserving data analysis by demonstrating the successful integration of differential privacy techniques into a production-ready data processing framework using Rust and WebAssembly. The outcomes highlight the potential for leveraging modern technologies to create portable, secure, and efficient solutions for protecting sensitive information while enabling valuable data-driven insights. By addressing the challenges and limitations identified in this discussion, future research can build upon this foundation to advance the state-of-the-art in differential privacy and promote its widespread adoption in real-world applications.