\chapter{User Manual\label{chap:user_man}}

\section{Quickstart guide}
\begin{enumerate}
    \item Extract the tarball, open your terminal interface and navigate to the extracted tarball.
    \item Ensure Docker or Podman is installed. If using Docker, ensure your user is either in your sudoers file or otherwise has permission to control the Docker daemon (rootless Docker).
    \item Run \texttt{podman build --tag=julien-honours .} \\
    Or \texttt{sudo docker build --tag=julien-honours .}
    \item Run \texttt{podman run -p 8888:8888 julien-honours} \\
    Or \texttt{sudo docker run -p 8888:8888 julien-honours} \\
    \textit{Note:} Ensure no other running service is running on TCP port 8888. Alter the above command to choose an alternative port mapping if applicable.
    \item Navigate to the URL printed in the terminal. You cannot directly connect to localhost:8888 without possessing the access token embedded into the URL.
    \item You will be presented with a Jupyter Notebook environment. Navigate to \texttt{project.ipynb} where you will find the results evaluated between the input and output datasets.
    \item To rerun the notebook, go to "Run">"Run All Cells" in Jupyter's toolbar
\end{enumerate}

\section{Build arguments and setup details}
Before being able to use the automated set-up and execution of this project, an OCI Engine is required to be set up and functional. Consult the documentation to install your OCI Engine of choice. This project has been validated with both Podman and Docker. Installation instructions for Podman can be found \href{https://podman.io/docs/installation}{here},  or Docker \href{https://docs.docker.com/get-docker/}{here}. Note that if using Docker, it may be necessary to run the below commands as root, unless a rootless Docker environment has been set-up. More information on rootless Docker \href{https://docs.docker.com/engine/security/rootless/}{here}. This project was primarily developed and validated using Podman.

The Dockerfile can use either a locally compiled version of Fluent Bit or one of their remote images. As their default images are distroless, and thus lack a shell, one of their "debug" images are required. It is recommended that if using a remote image that either: latest-debug, 2.2.2-debug, or 3.0.0-debug. At the time of writing, 3.0.2 was recently released with the plugin being primarily developed using 2.2.2. Since the release of 2.2.2, the WASM plugin infrastructure has seen additional commits, however, so far no backwards compatibility-breaking changes have been introduced. It therefor should be safe to run with the latest stable release. Using one of the remote images must be accompanied with --build-arg wasm\_optimization=wasm as the \texttt{flb-wamrc} binary is not compiled by default (which is the motivation to have a local build option, as \texttt{FLB\_WAMRC=On} is specified for CMake). 

The Dockerfile is set to forward a Rust profile selection for building the filter\_dp plugin. Use the rust\_profile build argument to control this selection. By default, the release profile is set; this is the recommended configuration. If the debug profile is specified, log messages will be sent to stdout. When using the debug profile, it is mandatory to use the AoT wasm\_optimization target. Failure to do so will give a "[error] Got exception running wasm code: Exception: wasm operand stack overflow" error. The script will forward any profile specified, however no profile-specific code is currently in place outside several println macros tied to the debug profile. Both rust\_profile wasm\_optimization arguments lack input validated and, therefore could be susceptible to code injection, however, this poses only a minor widening of the attack surface due to the level of access already required to exploit. If this script is anyway used in future in such a manner where user input is expected to be able to configure these build-time flags, proper input validations should be sought. 
