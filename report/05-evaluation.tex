\chapter{Evaluation\label{chap:evaluation}}
\section{Experimental Design}
To evaluate the Rust plugin, we utilized a sample dataset consisting of two CSV files: "EmployeeSalaries.csv" and "StudentsPerformance.csv". The "EmployeeSalaries.csv" file contains information about employee salaries, including attributes such as department, gender, base salary, and overtime pay. The "StudentsPerformance.csv" file contains data related to student performance, including attributes like gender, race/ethnicity, parental education, and test scores.
The experimental setup involved the following steps:

\begin{enumerate}
    \item The sample dataset files were split into smaller chunks to simulate a streaming data scenario. Each file was divided into subfiles containing 200 records each.
    \item The Rust plugin was configured with specific differential privacy settings for each attribute. For the \texttt{"Base\_Salary"} attribute in the "EmployeeSalaries.csv" file, Gaussian noise with an epsilon value of 1.0 and a sensitivity of 10 was applied. For the \texttt{"math\_score"} \texttt{"reading\_score"}, and \texttt{"writing\_score"} attributes in the \texttt{"StudentsPerformance.csv"} file, Laplace noise was used with different epsilon values and sensitivities.
    \item Fluent Bit, a lightweight data processor, was used to process the data streams. The Rust plugin was integrated into Fluent Bit as a filter, allowing for the application of differential privacy mechanisms to the data in real-time.
    \item The perturbed datasets were stored in separate output files, "EmployeeSalaries.perturbed.csv" and "StudentsPerformance.perturbed.csv", for further analysis.
\end{enumerate}
\section{Evaluation Metrics}
To assess the performance of the differential privacy mechanisms implemented in the filter plugin, we employ three commonly used evaluation metrics: Mean Absolute Error (MAE), Mean Squared Error (MSE), and Root Mean Squared Error (RMSE). Equations in this section were sourced from \cite{Hodson2022}
\subsection{Mean Absolute Error (MAE)}
The Mean Absolute Error (MAE) measures the average magnitude of the errors between the original and perturbed values, without considering their direction. It is calculated by taking the average of the absolute differences between the original and perturbed values \cite{Willmott2005}. The formula for MAE is given by:
\begin{equation}
MAE = \frac{1}{n} \sum_{i=1}^{n} |y_i - \hat{y}_i|
\end{equation}
where $n$ is the number of samples, $y_i$ is the original value, and $\hat{y}_i$ is the perturbed value.
\subsection{Mean Squared Error (MSE)}
The Mean Squared Error (MSE) measures the average squared difference between the original and perturbed values \cite{ZhouWang2009}. It is calculated by taking the average of the squared differences between the original and perturbed values. The formula for MSE is given by:
\begin{equation}
MSE = \frac{1}{n} \sum_{i=1}^{n} (y_i - \hat{y}_i)^2
\end{equation}
MSE gives more weight to larger errors due to the squaring operation, making it more sensitive to outliers compared to MAE.
\subsection{Root Mean Squared Error (RMSE)}
The Root Mean Squared Error (RMSE) is the square root of the Mean Squared Error \cite{Chai2014}. It is calculated by taking the square root of the average squared differences between the original and perturbed values. The formula for RMSE is given by:
\begin{equation}
RMSE = \sqrt{\frac{1}{n} \sum_{i=1}^{n} (y_i - \hat{y}_i)^2}
\end{equation}
RMSE has the same units as the original data, making it easier to interpret compared to MSE. It is also more sensitive to larger errors due to the squaring operation.
These evaluation metrics provide insights into the accuracy and utility of the differentially private data generated by the filter plugin. Lower values of MAE, MSE, and RMSE indicate better performance and higher data utility.
\subsection{Test Environment}
The test environment was constructed with a multi-build context Docker container. The first two stages concerned themselves with building the filter plugin along with Fluent Bit with a non-default build option (only needed if \acrshort{aot} is to be built).
\subsection{Experimental Setup}
\section{Results}
\section{Analysis}
\section{Interpretation of Results}