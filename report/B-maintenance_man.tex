\chapter{Maintenance Manual\label{chap:maintenance_man}}



The use of the software package may be limited to UNIX-like operating systems as \newline\mintinline{cmake}{set(FLB_FILTER_WASM           No)} is present in \mintinline[breaklines, breakafter=/.-]{bash}{fluent-bit/cmake/windows-setup.cmake} therefore as of the current release you cannot run WASM filter plugins on the Windows-native version of Fluent Bit without changing that build flag (and thus running with an unsupported change). As the provided Dockerfile will do all Fluent Bit operations inside of a Debian Bookworm (Slim) container image, the use of Docker-on-Windows should be functional.

\section{Using your host's Rust packages}
If you wish to develop the application on your host, you will need to make sure a modern version of rust and cargo is installed on your system and available to your applicable PATH. In addition, the wasm32-wasi target must be installed if you wish to produce a WASM binary. This can be accomplished by either using Rustup or through your Linux distribution's native packaging.
\begin{itemize}
    \item For Fedora and derivatives:\mintinline{bash}{dnf install rust rust-std-static-wasm32-wasi cargo}
    \item For Debian and derivatives:\mintinline{bash}{apt install rust-all libstd-rust-dev-wasm32} \break Note: Current Debian Stable release for libstd-rust-dev-wasm32 is 1.63.0 (similar goes for the rust-all package), all testing/development was done with the latest version of these packages which was packaged with Fedora and Rustup which both use the 1.77 series. Use Rustup if you wish to ensure compatibility.
    \item For any Unix-like system:
    \begin{itemize}
        \item \mintinline{bash}{curl --proto '=https' --tlsv1.2 -sSf https://sh.rustup.rs | sh}. Rustup is also packaged if you wish to avoid the security concerns that are inherently present when piping curl into sh.
        \item Ensure that wasm32-wasi target is installed when prompted to specify your desired default target. Alternatively, \mintinline{bash}{rustup target add wasm32-wasi} if rustup is already present on your host system. 
    \end{itemize}
\end{itemize}
However nothing of the code is WASM-spesific, so you can use the library compiled to any architecture which has support for all used dependencies. You are able to use the library from any programming language supporting the C Foreign Function Interface (thus, applicable C types). Note: wasm32-unknown-unknown is an unsupported target. We require the WASI environment to provide an interface for file I/O operations, in addition, rand requires manual intervention to specify the source of randomness where none can be assumed. The application is designed to panic if an improperly formatted UTF-8  

\section{Apple Silicon Mac patch}
In order to build on an Apple Silicon Mac, the following patch may be applied to the fluent-bit submodule. This is needed at least when running on Asahi Linux while using WAMR's AOT compile option. WAMR autodetects aarch64 when optimizing for the current architecture, however, when Fluent Bit attempts to load the binary it complains that it was expecting "aarch64v8" and received "aarch64". Going by the documentation, intended behaviour is that anything built for aarch64 should run on any aarch64 varient, however this may be a Fluent Bit configuration error.
\begin{minted}[
frame=lines,
framesep=2mm,
baselinestretch=1.2,
fontsize=\footnotesize,
linenos,
breaklines, breakafter=/.-]{c}
diff --git a/lib/wasm-micro-runtime-WAMR-1.3.0/core/iwasm/aot/arch/aot_reloc_aarch64.c b/lib/wasm-micro-runtime-WAMR-1.3.0/core/iwasm/aot/arch/aot_reloc_aarch64.cindex b4bb6024a..4593cd706 100644
--- a/lib/wasm-micro-runtime-WAMR-1.3.0/core/iwasm/aot/arch/aot_reloc_aarch64.c
+++ b/lib/wasm-micro-runtime-WAMR-1.3.0/core/iwasm/aot/arch/aot_reloc_aarch64.c
@@ -56,7 +56,7 @@ get_target_symbol_map(uint32 *sym_num)
 #if (defined(__APPLE__) || defined(__MACH__)) && defined(__arm64__)
 #define BUILD_TARGET_AARCH64_DEFAULT "arm64"
 #else
-#define BUILD_TARGET_AARCH64_DEFAULT "aarch64v8"
+#define BUILD_TARGET_AARCH64_DEFAULT "aarch64"
 #endif
 
 void
\end{minted}